By default, the LCD screen is always on. This allows you to see the temperature
from across the room without having to get up.

If you would prefer that the screen is off unless you are trying to adjust the
set points, there are two options: change some software settings so the screen
will be black with backlight still on, or upgrade the screen and make some
hardware modifications to have the backlight turn off entirely.

\subsubsection{Software modification}
To blank out the screen when not in use, you'll want to turn on Display Power
Management Signaling (DPMS). This can be done by logging in via SSH and running
\texttt{set +dpms} from the command line. After 10 minutes of not using the screen,
you should see the screen go black. In order to make sure this is set each time
the system reboots, add this command to \texttt{/etc/rc.local} before the
\texttt{exit 0} line at the end.

\subsubsection{Hardware modifications}
This is an advanced topic which requires familiarity with electronics,
soldering equipment, and a multimeter. Please read this entire section before
you start ordering parts or making modifications to be sure you will be
comfortable with making these changes.

First, you will need a Waveshare screen. The 3.5inch RPi LCD (B) and 3.5inch
RPi LCD (C) have both been tested and confirmed to be compatible. Please note
that the 3.5inch RPi LCD (A) does NOT have the ability to turn off the
backlight. The Waveshare documentation shows how to modify the screen in order
to enable the ability to control the backlight:
\url{https://www.waveshare.com/wiki/3.5inch\_RPi\_LCD\_(B)\#Control\_Backlight\_Brightness\_with\_GPIO}

Next comes the hard part: modifying the HestiaPi PCB. The LCD screen has 26
pins, all of which are connected to the the LCD header pins. However, pin 12
(GPIO 18) is used to control one of the relays (relay 3), which means this
can't be used to also control the LCD backlight.

The modifications that need to be made are to scrape the trace off the PCB that
connect pin 12 on the pi to pin 12 on the LCD as well as the trace between pin
12 of the LCD and relay 3. After these are no longer connected, you'll need to
solder a wire onto the PCB to reconnect pin 12 on the pi to relay 3. You'll
also need to connect pin 10 on the pi (GPIO 15) to pin 12 on the LCD. This will
allow GPIO 15 to control the LCD backlight. A future version of the PCB will
integrate this change, but unless your board has v2.53 or later printed on the
back, you will need to make these hardware modifications.

Now that the hardware modifications are done, we need to set up a script that
will actually turn off the backlight any time that the screen is blanked out by
the power management system. This will require downloading a shell script that
will run in the background. The script is called backlight.sh and can be
downloaded from the home/pi/scripts directory in
\url{https://github.com/HestiaPi/hestia-touch-openhab}. You will need to make
it executable with:

\texttt{chmod +x /home/pi/scripts/backlight.sh}.

This script will be included in the HestiaPi SD card image as of version 1.5.

Finally, we need to make sure this script is run when the pi boots up. To do
this, add the following line to \texttt{/etc/rc.local} before the
\texttt{exit 0} line:

\texttt{/home/pi/scripts/backlight.sh \&}

Note the ampersand at the end of the line. This is important, as it is what
causes the script to be run in the background.
